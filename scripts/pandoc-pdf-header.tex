% scripts/pandoc-pdf-header.tex — Sans-serif, figures stables, code lisible

\usepackage{ifxetex,ifluatex}
\ifxetex
  \usepackage{fontspec}
  \IfFileExists{texgyre.sty}{}{}
  \setmainfont{TeX Gyre Heros}
  \setsansfont{TeX Gyre Heros}
  \setmonofont{DejaVu Sans Mono}
\else\ifluatex
  \usepackage{fontspec}
  \setmainfont{TeX Gyre Heros}
  \setsansfont{TeX Gyre Heros}
  \setmonofont{DejaVu Sans Mono}
\else
  \usepackage[T1]{fontenc}
  \usepackage[scaled]{helvet}
  \renewcommand{\familydefault}{\sfdefault}
\fi\fi

% Couleurs
\usepackage{xcolor}
\definecolor{accent}{HTML}{0056B3}
\definecolor{accenttwo}{HTML}{2C3E50}
\definecolor{rules}{HTML}{D8DEE9}
\definecolor{muted}{HTML}{6B7280}
\definecolor{codebg}{HTML}{F8F9FA}

% Mise en page & images
\usepackage[a4paper,left=1.5cm,right=1.5cm,top=1.5cm,bottom=1.5cm]{geometry}
\usepackage{microtype}
\setlength{\emergencystretch}{2em}
\usepackage{graphicx}
\setkeys{Gin}{width=\linewidth, keepaspectratio}
% Personnalisation manuelle des légendes (sans babel/polyglossia)
\renewcommand{\figurename}{Figure}
% Ajouter l'espace fine avant les deux-points dans les légendes
\usepackage{caption}
\DeclareCaptionLabelSeparator{frenchcolon}{~:\space}
\captionsetup{labelsep=frenchcolon}

% --- FIGURES : empêcher les remontées vers la page de ToC et couper proprement
\usepackage{float}                  % pour [H]
\floatplacement{figure}{H}          % place les figures "ici"
\usepackage[section]{placeins}      % \FloatBarrier auto à chaque \section
\setlength{\intextsep}{12pt}
\setlength{\textfloatsep}{16pt}

% Liens / PDF
\usepackage{hyperref}
\hypersetup{ colorlinks=true, linkcolor=red, urlcolor=magenta, citecolor=green, linktoc=all }

% Titres
\usepackage{titlesec}
\titleformat{\section}{\Large\bfseries\color{accent}}{\thesection}{0.6em}{}[\color{accent}\titlerule]
\titlespacing*{\section}{0pt}{16pt}{8pt}
\newcommand{\leftdecor}{\textcolor{rules}{\rule{4pt}{1.2em}}\hspace{0.6em}}
\titleformat{\subsection}[hang]{\large\bfseries\color{accenttwo}}{\leftdecor\thesubsection}{0.6em}{}
\titlespacing*{\subsection}{0pt}{14pt}{6pt}
\newcommand{\leftdecorsmall}{\textcolor{rules}{\rule{3pt}{1em}}\hspace{1.2em}}
\titleformat{\subsubsection}[hang]{\normalsize\bfseries\color{accenttwo}}{\leftdecorsmall\thesubsubsection}{0.6em}{}
\titlespacing*{\subsubsection}{0pt}{12pt}{4pt}

% Table des matières (style)
\usepackage{tocloft}
\renewcommand{\cftsecleader}{\cftdotfill{\cftdotsep}}
\setlength{\cftbeforesecskip}{2pt}
\setlength{\cftbeforesubsecskip}{1pt}
\setlength{\cftbeforesubsubsecskip}{0pt}
\renewcommand{\cftsecfont}{\bfseries}
\renewcommand{\cftsubsecindent}{1.5em}
\renewcommand{\cftsubsubsecindent}{3em}
\renewcommand{\cftsubsubsecfont}{\color{muted}}

% Listes
\usepackage{enumitem}
\setlist{itemsep=3pt, topsep=4pt, parsep=2pt, partopsep=0pt}

% Support des colonnes multiples
\usepackage{multicol}

% Code (listings) — nécessite l’option pandoc --listings
\usepackage{listings}
\lstset{
  basicstyle=\ttfamily\small,
  backgroundcolor=\color{codebg},
  frame=single,
  rulecolor=\color{rules},
  framesep=6pt,
  xleftmargin=0pt,
  aboveskip=6pt,
  belowskip=6pt,
  columns=fullflexible,
  keepspaces=true,
  showstringspaces=false,
  breaklines=true,
  breakatwhitespace=false,
  breakautoindent=true,
  postbreak=\mbox{\textcolor{rules}{$\hookrightarrow$}\space}
}

% --- Préambule requis ---
\usepackage{xcolor}
\usepackage[most]{tcolorbox}
\tcbuselibrary{breakable,skins}
\usepackage{setspace}

% Couleurs (adapte si besoin)
\definecolor{quotebg}{gray}{0.98}      % fond gris clair
% \definecolor{rules}{HTML}{005BBB}     % si 'rules' n'est pas déjà défini ailleurs

% Boîte "quote" avec barre gauche, fond, padding, italique + interligne 1.3
\newtcolorbox{quoteblock}{
  enhanced,
  breakable,
  boxrule=0pt,                              % pas de bordure complète
  colback=quotebg,                          % fond
  coltext=black,                            % texte
  borderline west={4pt}{0pt}{accent},        % barre verticale gauche (couleur 'accent')
  left=1em, right=2em, top=1em, bottom=0.6em, % padding interne
  before skip=\smallskipamount,             % espacements externes
  after  skip=\smallskipamount,
  fontupper=\itshape,                       % texte en italique (retire si tu veux)
  before upper=\begin{spacing}{1.3},        % interligne augmenté
  after  upper=\end{spacing}
}

% Redéfinition de l'environnement "quote"
\renewenvironment{quote}{\begin{quoteblock}}{\end{quoteblock}}

% Environnement pour citations en colonnes
\newenvironment{quotecols}[1][0.48]{%
  \par\noindent
  \begin{minipage}[t]{#1\textwidth}
}{%
  \end{minipage}
}

% Citation sans breakable pour colonnes
\newtcolorbox{quotecolblock}{
  enhanced,
  % PAS de breakable ici
  boxrule=0pt,
  colback=quotebg,
  coltext=black,
  borderline west={4pt}{0pt}{accent},
  left=1em, right=2em, top=1em, bottom=0.6em,
  before skip=\smallskipamount,
  after skip=\smallskipamount,
  fontupper=\itshape,
  before upper=\begin{spacing}{1.3},
  after upper=\end{spacing}
}

\newenvironment{quotecol}{\begin{quotecolblock}}{\end{quotecolblock}}

% Définition des couleurs Swagger
\definecolor{swaggerblue}{HTML}{61AFFE}
\definecolor{swaggergreen}{HTML}{49CC90}
\definecolor{swaggerpurple}{HTML}{9012FE}
\definecolor{swaggerorange}{HTML}{FCA130}
\definecolor{swaggerred}{HTML}{F93E3E}

% Macros pour les méthodes HTTP
\newcommand{\httpget}{\textcolor{swaggerblue}{\textbf{GET}}}
\newcommand{\httppost}{\textcolor{swaggergreen}{\textbf{POST}}}
\newcommand{\httpput}{\textcolor{swaggerpurple}{\textbf{PUT}}}
\newcommand{\httppatch}{\textcolor{swaggerorange}{\textbf{PATCH}}}
\newcommand{\httpdelete}{\textcolor{swaggerred}{\textbf{DELETE}}}

% Veuves/orphelines
\clubpenalty=10000
\widowpenalty=10000
\displaywidowpenalty=10000

% Force la configuration des liens à la fin
\AtBeginDocument{%
    \hypersetup{
        colorlinks=true,
        linkcolor=accent,
        urlcolor=accent,
        citecolor=accent,
        linktoc=all
    }%
}

% --- Pied de page "Page x / total" ---
\usepackage{fancyhdr}
\usepackage{lastpage} % doit être chargé (souvent) après hyperref pour \pageref{LastPage}
\pagestyle{fancy}
\fancyhf{} % vide tout
% Pied centré :
\fancyfoot[C]{\small\color{gray}Page \thepage{} / \pageref*{LastPage}}
% Pas de ligne d'en-tête/pied :
\renewcommand{\headrulewidth}{0pt}
\renewcommand{\footrulewidth}{0pt}

% CORRECTION : Forcer fancy pour la TOC et toutes les pages
\fancypagestyle{plain}{%
  \fancyhf{}%
  \fancyfoot[C]{\small\color{gray}Page \thepage{} / \pageref*{LastPage}}
  \renewcommand{\headrulewidth}{0pt}%
  \renewcommand{\footrulewidth}{0pt}%
}

% S'assurer que fancy est appliqué dès le début
\AtBeginDocument{\pagestyle{fancy}}
% Optionnel : laisser la page de garde sans numéro si tu utilises \begin{titlepage}
% (tu as déjà \thispagestyle{empty} dans la page de garde, donc c'est bon)
